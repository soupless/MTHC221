% Sets and relations
% 01/17/2024

\section{Sets and relations}
    \renewcommand{\leftmark}{January 17, 2024}
    \begin{dfn}[Cartesian product]
        Let \(A\) and \(B\) be sets. The \emph{Cartesian product} of \(A\) and \(B\), denoted by \(A \times B\), is defined as \[A \times B \ :=\  \{(a, b) \,|\: a\in A,\, b\in B\}.\]\vspace{-1ex}
    \end{dfn}

    \begin{dfn}[Relation]
        A relation \(R\) between sets \(A\) and \(B\) is a subset of \(A \times B\). That is, \(R \subseteq A \times B\).
    \end{dfn}

    We let \(aRb \equiv (a, b)\in\real\) for each \(a\in A\) and \(b\in B\) where \(R\) is a relation. It is an implicit assumption that \(A\) and \(B\) have to be nonempty, otherwise, the relation is trivial.

    \marker{thm:number-of-relations}
    \begin{thm}
        If \(A\) and \(B\) are finite sets, then there are \(2^{|A| \cdot |B|} - 1\) relations.
    \end{thm}

    \begin{proof}
        Let \(A\) and \(B\) be finite sets. Then, the question is equivalent to finding the number of subsets of \(A \times B\), which is \(\left|\pset{A \times B}\right| - 1 = 2^{|A \times B|} - 1 = 2^{|A| \cdot |B|} - 1 \).
    \end{proof}

    \begin{dfn}[Function]
        A function \(\phi\) mapping \(X\) into \(Y\) is a relation between \(X\) and \(Y\) with the property that each \(x\in X\) appears as the first member of exactly one ordered pair \((x,y)\in\phi\) for all \(y\in Y\).
    \end{dfn}

    \begin{dfn}[Domain, codomain, range]
        Let \(\phi : X \to Y\) be a function mapping \(X\) to \(Y\). Then, 
        \begin{itemize}
            \item \(X\) is the domain of \(\phi\),
            \item \(Y\) is the codomain of \(\phi\),
            \item \(\phi[X]\) is the range of \(\phi\) such that \(\phi[X] = \{\phi(x) \,\mid\, x\in X\}\).
        \end{itemize}

        Another notation would be \(X \overset{\phi}{\longrightarrow} Y\) to denote the type signature, and \(x \overset{\phi}{\longmapsto} y\) to denote the function definition.
    \end{dfn}

    \begin{dfn}[Injective function]
        A function \(\phi : X \to Y\) is \emph{injective} or one-to-one (1-1) if, for all elements \(x_1\) and \(x_2\) of \(X\), \(\phi(x_1) = \phi(x_2)\) implies \(x_1 = x_2\).
    \end{dfn}

    \begin{example}
        Let \(f : \real\to\real\) such that \(f(x) = 2x + 3\) for all \(x\in\real\). Is \(f\) injective?

        Let \(x_1\) and \(x_2\) be arbitrary elements of \(\real\), and suppose that \(f(x_1) = f(x_2)\). Then,
        \begin{align*}
            f(x_1) &= f(x_2) \\
            2x_1 + 3 &= 2x_2 + 3 \\
            2x_1 &= 2x_2 \\
            x_1 &= x_2
        \end{align*}
        Hence, \(f\) is injective.
    \end{example}

    \begin{example}
        Let \(g : \real\to\real\) such that \(g(x) = x^2\) for all \(x\in\real\). Is \(g\) injective?

        No, because \(g(-1) = 1\), and \(g(1) = 1\), but \(-1 \neq 1\).
    \end{example}

    \begin{dfn}[Surjective function]
        A function \(\phi : X \to Y\) is surjective or onto if \(\phi[X] = Y\). Equivalently, \(\forall y\!\in\! Y\  \exists x\!\in\! X (\phi(x) = y)\).
    \end{dfn}

    \begin{example}
        Let \(F : \real\to\real\) defined by \(F(x) = x^2\) for each \(x\in\real\). Is \(F\) surjective?

        Since \(F(x) \geq 0\) for all \(x\in\real\), then \(F(x) = -1\) has no solution. Hence, \(F\) is not surjective.
    \end{example}

    \begin{example}
        Let \(G : \nat \to \nat\) such that \(G(x) = x + 1\) for each \(x\in\nat\). We define \(\nat = \{1, 2, 3, \ldots\}\). Is \(G\) surjective?

        No, because \(G(x) = 1\) has no solution in \(\nat\). Hence, \(G\) is not surjective.
    \end{example}

    \begin{example}
        Let \(\phi : \nat\to\nat\) such that \(\phi(n)\) is the \(n\)th prime number for each \(n\in\nat\). Then, \(\phi\) is injective. However, it is not surjective, because \(\phi(x) = 4\) has no solution.
    \end{example}

    \begin{example}
        Let \(g : \real\to\real\) such that \(g(x) = x + 1\) for each \(x\in\real\). Prove that \(g\) is surjective.

        Let \(y\in\real\) be arbitrary. Then, \(g(x) = y \implies x + y = 1 \implies x = y - 1\). Since \(g(y - 1) = y - 1 + 1 = y\), this means \(g\) is surjective.
    \end{example}

    \begin{dfn}[Bijective function]
        If \(\phi : X \to Y\) is both injective and surjective, then \(\phi\) is bijective.
    \end{dfn}

    \begin{dfn}[Inverse]
        Let \(\phi : X \to Y\) be a bijective function. The inverse of \(\phi\), denoted by \(\phi^{-1}\), is the function \(\phi^{-1} : Y \to X\) such that \(\phi^{-1}(y) = x \iff \phi(x) = y\) for all \(x\in X\) and \(y\in Y\).
    \end{dfn}

    A representation for finite domain maps would be through a matrix representation like \[\phi : \begin{bmatrix}
        x_1 & x_2 & \cdots & x_n \\
        \phi(x_1) & \phi(x_2) & \cdots & \phi(x_n)
    \end{bmatrix}\]

    \begin{dfn}[Function composition]
        Let \(\phi : A \to B\) and \(\theta : B \to C\) be functions. The composition \(\theta\phi\) is the function \(\theta\phi : A \to C\) defined by \(\theta\phi(a) = \theta\left(\phi(a)\right)\) for each \(a\in A\).
    \end{dfn}

    \begin{dfn}[Function equality]
        Let \(f : X \to Y\) and \(g : X \to Y\). Then, \(f = g\) if \(\forall x \in X \left(f(x) = g(x)\right)\).
    \end{dfn}

    \marker{thm:fcprop}
    \begin{thm}
        Given functions \(\alpha : A \to B\), \(\beta : B \to C\), and \(\gamma : C \to D\), then:
        \begin{enumerate}
            \item \((\gamma\beta)\alpha = \gamma(\beta\alpha)\). That is, function composition is associative.

            \item If \(\alpha\) and \(\beta\) are both injective, then \(\beta\alpha\) is injective.

            \item If \(\alpha\) and \(\beta\) are both surjective, then \(\beta\alpha\) is also surjective.
        \end{enumerate}
    \end{thm}

    \begin{proof} We prove each property listed in \hyperref[thm:fcprop]{the theorem}:
        \begin{itemize}
            \item Suppose we have the functions \(\alpha : A \to B\), \(\beta : B \to C\), and \(\gamma : C \to D\). Let \(a\in A\) be arbitrary. Then,
            \begin{align*}
                (\gamma\beta)\alpha(a) &= \gamma\beta\left(\alpha(a)\right) \\
                (\gamma\beta)\alpha(a) &= \gamma\left(\beta\left(\alpha(a)\right)\right) \\\\
                \gamma(\beta\alpha)(a) &= \gamma\left(\beta\alpha(a)\right) \\
                \gamma(\beta\alpha)(a) &= \gamma\left(\beta\left(\alpha(a)\right)\right)
            \end{align*}
            Hence, \((\gamma\beta)\alpha(a) = \gamma(\beta\alpha)(a)\). Therefore, \((\gamma\beta)\alpha = \gamma(\beta\alpha)\).

            \item Let \(\alpha : A \to B\) and \(\beta : B \to C\) be injective functions. Then, \(\beta\alpha : A \to C\). Suppose that for all \(a_1, a_2 \in A\), \(\beta\alpha(a_1) = \beta\alpha(a_2)\). We get the following derivation:
            \begin{align*}
                \beta\alpha(a_1) &= \beta\alpha(a_2) \\
                \beta\left(\alpha(a_1)\right) &= \beta\left(\alpha(a_2)\right) \\
                \alpha(a_1) &= \alpha(a_2) \\
                a_1 &= a_2
            \end{align*}
            Therefore, \(\beta\alpha\) is injective.

            \item Let \(\alpha : A \to B\) and \(\beta : B \to C\) be surjective functions. Let \(c \in C\) be arbitrary. Then, there exists \(b \in B\) such that \(\beta(b) = c\). Since \(\alpha\) is surjective, there exists \(a\in A\) such that \(\alpha(a) = b\). Then, \(\beta(b) = \beta(\alpha(a)) = \beta\alpha(a) = c\). Hence, there exists \(a\in A\) such that \(\beta\alpha(a) = c\), and so for all \(c \in C\), there exists \(a\in A\) such that \(\beta\alpha(a) = c\). Therefore, \(\beta\alpha\) is surjective.
        \end{itemize}
    \end{proof}
