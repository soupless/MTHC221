    \section{Cyclic groups, continued}
    \renewcommand{\leftmark}{April 01, 2024}

    \begin{thm}[Fundamental Theorem of Cyclic Groups]
        Every subgroup of a cyclic group is cyclic. Moreover, if \(|\langle a \rangle| = n\), then the order of any subgroup of \(\langle a \rangle\) is a divisor of \(n\), and for each positive divisor \(k\) of \(n\), the group \(\langle a \rangle\) has exactly one subgroup of order \(k\), namely \(\langle a^{n/k} \rangle\).
    \end{thm}

    \begin{proof}
        We prove the theorem by parts:

        \begin{itemize}
            \item We first show that any subgroup of a cyclic group is cyclic. Let \(G\) be a cyclic group. Then, \(G = \langle a \rangle\). Let \(H \leq G\). If \(H\) is generated by \(e\), then \(H\) is trivially cyclic. Suppose \(H\) is not generated by \(e\). There exists an element of the form \(a^t\) in \(H\) such that \(a^t \neq e\) for some \(t > 0\). Since \(a^t \in H\) iff \(a^{-t} \in H\), choose \(\min \{t : a^t \in H, t > 0\}\). Let \(m\) be this number. We claim that \(H = \langle a^m \rangle\).

            \item Clearly, \(\langle a^m \rangle \subseteq H\) by closure. We show that \(H \subseteq \langle a^m \rangle \). Let \(a^k \in H\). By the division algorithm, there exists a \(0 \leq r < m\) such that \(k = mq + r\). Then, \(a^{k} = a^{mq + r}\), so \(a^{k}a^{-mq} = a^{r}\). Observe that \(a^k \in H\) and \(a^{-mq} \in H\), so \(a^{r} \in H\). If \(0 < r < m\), then we get a contradiction since we have \(m\) as the minimal exponent. Hence, \(r = 0\). Thus, \(a^{k} = a^{mq} = (a^{m})^{q}\), so \(a^{k} \in \langle a^{m}\rangle\). Hence, \(H \subseteq \langle a^{m}\rangle\), so \(H = \langle a^{m}\rangle\).

            \item Let \(|\langle a \rangle| = n\), and let \(k\) be a positive divisor of \(n\). We show that there is a unique subgroup of order \(k\), namely \(\langle a^{n/k} \rangle\).

            We first show existence. Consider \(\langle a^{n/k} \rangle\). Then, \(|\langle a^{n/k}\rangle| = |a^{n/k}| = n/\gcd(n, n/k) = n/(n/k) = k\). We now show uniqueness. Let \(H \leq \langle a\rangle\) such that \(|H| = k\). Then, there exists an \(m\in \int^+\) such that \(H = \langle a^{m}\rangle\). This means \(|H| = |a^m| = n/\gcd(n, m) = n/m\). Since \(|H| = k\), then \(k = n/m\), so \(m = n/k\). Therefore, the subgroup is unique.
        \end{itemize}
    \end{proof}

    A corollary would about subgroups of \(\int_n\). For each positive divisor \(k\) of \(n\), the group \(\langle n/k \rangle\) is the unique subgroup of order \(k\).

    \begin{thm}
        If \(d\) is a positive divisor of \(n\), the number of elements of order \(d\) is \(\phi(d)\).
    \end{thm}

    % TODO: Check for continuation