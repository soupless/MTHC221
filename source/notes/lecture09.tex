\section{Cyclic groups}
    \renewcommand{\leftmark}{March 06, 2024}

    \begin{dfn}[Cyclic groups]
        Let \(G\) be a group. \(G\) is cyclic if there exists \(g\in G\) such that \(G = \langle g\rangle\). We say that \(g\) is called a generator of \(G\). If \(G\) has no generator, then \(G\) is called acyclic.
    \end{dfn}

    \begin{example}
        Consider \(U_9\). The subgroups generated by each element are the following:
        \begin{align*}
            \langle 1\rangle &= \{1\} \\
            \langle 2\rangle &= \{2, 4, 8, 7, 5, 1\} \\
            \langle 4\rangle &= \{4, 7, 1\} \\
            \langle 5\rangle &= \{5, 7, 8, 4, 2, 1\} \\
            \langle 7\rangle &= \{7, 4, 1\} \\
            \langle 8\rangle &= \{8, 1\} 
        \end{align*}

        This means that the generators of \(U_9\) are 2 and 5.
    \end{example}

    \begin{example}
        The infinite group \(\int\) is cyclic, since 1 generates \(\int\).
    \end{example}

    \begin{thm}
        Let \(G\) be a group and let \(a \in G\). If \(a\) has infinite order, then \(a^i = a^j\) iff \(i = j\).
    \end{thm}

    \begin{proof}
        Let \(G\) be a group and let \(|a| = \infty\).

        \begin{itemize}
            \item[(\(\Leftarrow\))] Trivial.
            \item[(\(\Rightarrow\))] Suppose \(a^i = a^j\). Without loss of generality, let \(i > j\). Then,
            \begin{align*}
                a^i &= a^j \\
                a^ia^{-j} &= e \\
                a^{i - j} &= e \\
            \end{align*}
            Hence, \(|a| \leq i - j\), contradicting our assumption that the order of \(a\) is infinite. Therefore, \(i = j\).
        \end{itemize}
    \end{proof}

    \begin{thm}
        Let \(G\) be a group and let \(a\in G\). If \(|a| = n < \infty\), then \(\langle a \rangle = \{e, a, a^2, a^3, \ldots, a^{n - 1}\}\).
    \end{thm}

    \begin{proof}
        Suppose \(|a| = n\). Clearly, \(\{e, a, a^2, \ldots, a^{n -1}\} \subseteq \langle a \rangle\) since \(a^k \in \langle a \rangle\) for any \(k\in\int\).

        To show the converse, let \(a^k \in \langle a \rangle\). Using the division algorithm on \(k\), there exists nonnegative integers \(q, r\) such that \(k = nq + r\) where \(0 \leq r \leq n\). Then,
        \begin{align*}
            a^k &= a^{nq + r} \\
            a^k &= a^{nq}a^r \\
            a^k &= (a^{n})^q a^r \\
            a^k &= (e)^q a^r \\
            a^k &= e a^r \\
            a^k &= e a^r \\
            a^k &= a^r \in \{e, a, a^2, a^3, \ldots, a^{n - 1}\} \\
            a^k &\in \{e, a, a^2, a^3, \ldots, a^{n - 1}\}
        \end{align*}

        Now, all elements in \(\{e, a, \ldots, a^{n - 1}\}\) must be unique. If not, then there must be integers \(s, t\), without loss of generality, \(s < t\), such that \(a^t = a^s\) when \(0 \leq s, t < n\). This means \(a^{t - s} = e\), so \(|a| \leq t - s\) which contradicts the fact that \(t - s < n\).

        Hence, \(\langle a \rangle \subseteq \{e, a, a^2, a^3, \ldots, a^{n - 1}\}\). Therefore, \(\langle a \rangle = \{e, a, a^2, a^3, \ldots, a^{n - 1}\}\)
    \end{proof}
    
    \begin{thm}
        \marker{thm:div-order}
        Let \(G\) be a group and let \(a \in G\). If \(|a| = n < \infty\), then \(a^i = a^j\) iff \(n \divs i - j\).
    \end{thm}

    \begin{proof} We prove the two directions as follows:
        \mbox{}
        \begin{itemize}
            \item[\((\implies)\)] Suppose \(a^{i} = a^{j}\). Then, \(a^{i - j} = e\). By the division algorithm, there exists integers \(q,r\) such that \(i - j = nq + r\) where \(0 \leq r < n\). Then, \(a^{i - j} = a^{nq + r} = a^r = e\). If \(0 < r\), then \(|a| \leq r\) which implies \(n \leq r\), contradicting \(r < n\). This forces \(r\) to be zero. Hence, \(i - j = nq\), and so \(n \divs i - j\).

            \item[\((\Longleftarrow)\)] Suppose \(n \divs i - j\). There exists an integer \(k\) such that \(i - j = nk\). Then,
            \begin{align*}
                a^{i - j} &= a^{nk} \\
                a^{i - j} &= (a^n)^k \\
                a^{i - j} &= e^k \\
                a^{i - j} &= e \\
                a^{i - j}a^j &= ea^j \\
                a^i &= a^j
            \end{align*}
        \end{itemize}
    \end{proof}

    \begin{cor}
        \marker{cor:same-order-element-generator}
        For any group element \(a\), \(|a| = |\langle a\rangle|\).
    \end{cor}

    \begin{proof}
        We consider two cases, one where the order of \(a\) is finite and the other is infinite.
        \begin{itemize}
            \item The order is finite.

            Suppose \(|a| = n\). Since \(\langle a\rangle = \{e, a, \ldots, a^{n - 1}\}\) and \(|\langle a\rangle| = n\), then \(|a| = |\langle a\rangle|\).

            \item The order is infinite.

            Since \(\langle a\rangle = \{a^{i} \,\mid\, i \in\int\}\), then \(|\langle a\rangle| = |\int| = \infty\) which implies \(|a| = |\langle a\rangle|\).
        \end{itemize}
    \end{proof}

    \begin{cor}
        Let \(G\) be a group and let \(a \in G\) with \(|a| = n\). If \(a^k = e\), then \(n \divs k\) for some \(k\in\int\).
    \end{cor}

    \begin{proof}
        Let \(G\) be a group and let \(a \in G\) with \(|a| = n\). Since \(a^k = e\), then \(a^k = a^0\). This is iff \(n \divs k - 0\), which simplifies \(n \divs k\).
    \end{proof}

    \begin{thm}
        \marker{thm:order-of-power}
        Let \(a\) be a group element of order \(n\) and let \(k\) be a particular integer. Then \(\langle a^k\rangle = \langle a^{\gcd(n,k)}\rangle\) and \(|a^k| = n/\gcd(n, k)\).
    \end{thm}

    \begin{proof}
        Let \(a\) be a group element of \(G\) such that \(|a| = n\). Let \(d = \gcd(k, n)\). First, we show that \(\langle a^k\rangle = \langle a^d\rangle = \langle a^{\gcd(n,k)}\rangle\). We show \(\langle a^d\rangle \subseteq \langle a^k \rangle\).

        Note that for all integers \(i\), \(a^{ki} = a^{ds}\) for some integer \(s\). Since \(d = \gcd(n, k)\), by B\'ezout's lemma, there exists integers \(s, t\) such that \(ns + kt = d\). Then,
        \begin{align*}
            a^d &= a^{ns + kt} \\
            a^d &= a^{ns}a^{kt} \\
            a^d &= (a^n)^s a^{kt} \\
            a^d &= e^s a^{kt} \\
            a^d &= a^{kt}
        \end{align*}

        Since \(a^{kt} \in \langle a^k\rangle\), this means \(\langle a^d\rangle \subseteq \langle a^k\rangle\).

        Now, we show that \(\langle a^k\rangle \subseteq \langle a^d\rangle\). Let \(x \in \langle a^k\rangle\). Then, \(x\) will be of the form \(a^{ki}\) for some integer \(i\). Since \(d \divs k\), then \(k = dj\) for integer \(j\). This means that \(x = a^{ki} = a^{dji} = (a^d)^{ji}\) which implies \(x \in \langle a^{d}\rangle\). Hence, \(\langle a^k\rangle = \langle a^d\rangle = \langle a^{\gcd(n, k)} \rangle\).

        To prove that \(|a^k| = n/\gcd(n, k)\), we find the order of \(a^k\). It is the smallest positive integer \(m\) such that \((a^k)^m = e\). Since \(n\) is the smallest positive integer such that \(a^n = e\), then it must be that \(n \divs km - n\) (by Theorem \ref{thm:div-order}) which implies \(n \divs km\). We then get \[\frac{n}{\gcd(n, k)} \,\bigg|\, \frac{k}{\gcd(n, k)}m\] and since
        \begin{align*}
            \gcd(n, k) &= \gcd(n, k) \\
            \gcd\left(\frac{n}{\gcd(n, k)}, \frac{k}{\gcd(n, k)}\right) &= 1,
        \end{align*}
        then by Euclid's lemma, \(n/\gcd(n, k) \divs m\). Hence, \(n/\gcd(n, k) \leq m\). The smallest value of \(m\) satisfying the inequality is if \(m = n/\gcd(n, k)\). Hence, \(|a^k| = n/\gcd(n, k)\).
    \end{proof}

    \begin{example}
        Suppose that \(|a| = 100\). Find \(|a^{28}|\).

        Using the previous theorem,
        \begin{align*}
            |a^{28}| &= \frac{|a|}{\gcd(|a|, 28)} \\
            |a^{28}| &= \frac{100}{\gcd(100, 28)} \\
            |a^{28}| &= \frac{100}{4} \\
            |a^{28}| &= 25
        \end{align*}
    \end{example}

    \begin{example}
        Suppose \(|a| = 12\). Find \(|a^8|\) and \(\langle a^8\rangle\).

        Using the previous theorem, 
        \begin{align*}
            |a^8| &= \frac{|a|}{\gcd(|a|, 8)} & \langle a^8\rangle &= \langle a^{\gcd(12, 8)}\rangle \\ 
            |a^8| &= \frac{12}{\gcd(12, 8)} & \langle a^8\rangle &= \langle a^{4}\rangle \\ 
            |a^8| &= \frac{12}{4} & \langle a^8\rangle &= \langle a^4, a^8, e\rangle \\
            |a^8| &= 3
        \end{align*}
    \end{example}