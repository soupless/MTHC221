% Mess I made because I don't know moving arguments

\renewcommand{\sectionmark}[1]{\markboth{}{Quiz \thequizcounter}}
\titleformat{\section}{\Large\bfseries}{Quiz~\thequizcounter}{1ex}{}
\refstepcounter{quizcounter}
\section[Quiz \thequizcounter]{}\normalsize\titleformat{\section}{\Large\bfseries}{Lecture~\thesection:}{1ex}{}
\renewcommand{\sectionmark}[1]{\markboth{}{Lecture \thesection: #1}}
\vspace{-0.2in}
\renewcommand{\leftmark}{February 05, 2024}

% Quiz #1
% 02/05/2024

\begin{enumerate}
    \item Let \(S\) be the set of rational numbers and let \(R\) be a relation on \(S\) defined by \(aRb\) if \(ab \geq 0\). Is \(R\) an equivalence relation? Justify your answer. (10 points)

    \(R\) is not an equivalence relation, because it doesn't satisfy transitivity. Consider \(a,b,c\in S\) and suppose \(aRb\) and \(bRc\). Then, \(ab \geq 0\) and \(bc \geq 0\), so \(ab^2c \geq 0\). We cannot deduce \(ac \geq 0\) by dividing \(b^2\), since it is possible for \(b\) to be zero. Hence, it doesn't follow that \(aRc\) from \(aRb\) and \(bRc\).

    \item Let \(\sim\) be a relation on \(S\), where \(S\) is the set of real numbers, defined by \(a \sim b\) if \(a - b\) is an integer.
    \begin{enumerate}
        \item[(i)] Show that \(\sim\) is an equivalence relation on \(S\). (10 points)

        The relation \(\sim\) is an equivalence relation on \(S\) because it satisfies the requirements needed:
        \begin{itemize}
            \item Reflexivity

            For any given \(a\in S\), \(a - a = 0\in S\). Hence, \(a \sim a\).

            \item Symmetry

            Let \(a,b\in S\) be arbitrary, and suppose that \(a \sim b\). Then, \(a - b\) is an integer. And so is \(-(a - b)\), which is equivalent to \(b - a\). Hence, \(b \sim a\).

            \item Transitivity

            Let \(a,b,c\in S\) be arbitrary, and suppose that \(a \sim b\) and \(b \sim c\). Then, \(a - b\) and \(b - c\) are integers. And so is \(a - b + b - c\), which is equivalent to \(a - c\). Hence, \(a \sim c\).
        \end{itemize}

        \item[(ii)] Describe the equivalence class containing \(\sqrt{2}\). (5 points)

        Let \(x\in S\) be arbitrary. For \(\sqrt{2} \sim x\) to be true, \(\sqrt{2} - x\) needs to be an integer. This would only happen if \(\sqrt{2}\) vanishes, and what remains is an integer. Hence, \(x\) will be of the form \(-\sqrt{2} + k\) where \(k\in \int\). Therefore, \[
            [\sqrt{2}] = \{-\sqrt{2} + k \,\mid\, k\in\int\}.
        \]

    \end{enumerate}
    \item Construct the group table of the following: (10 points each)
    \begin{enumerate}
        \item[a.] \(\int_7\)

        \[
            \begin{array}{c|c|c|c|c|c|c|c|}
                * & 0 & 1 & 2 & 3 & 4 & 5 & 6 \\ \hline
                0 & 0 & 1 & 2 & 3 & 4 & 5 & 6 \\ \hline
                1 & 1 & 2 & 3 & 4 & 5 & 6 & 0 \\ \hline
                2 & 2 & 3 & 4 & 5 & 6 & 0 & 1 \\ \hline
                3 & 3 & 4 & 5 & 6 & 0 & 1 & 2 \\ \hline
                4 & 4 & 5 & 6 & 0 & 1 & 2 & 3 \\ \hline
                5 & 5 & 6 & 0 & 1 & 2 & 3 & 4 \\ \hline
                6 & 6 & 0 & 1 & 2 & 3 & 4 & 5 \\ \hline
            \end{array}
        \]

        \item \(U_{24}\)

        \[
            \begin{array}{c|c|c|c|c|c|c|c|c|}
                * & 1 & 5 & 7 & 11 & 13 & 17 & 19 & 23 \\ \hline
                1 & 1 & 5 & 7 & 11 & 13 & 17 & 19 & 23 \\ \hline
                5 & 5 & 1 & 11 & 7 & 17 & 13 & 23 & 19 \\ \hline
                7 & 7 & 11 & 1 & 5 & 19 & 23 & 13 & 17 \\ \hline
                11 & 11 & 7 & 5 & 1 & 23 & 19 & 17 & 13 \\ \hline
                13 & 13 & 17 & 19 & 23 & 1 & 5 & 7 & 11 \\ \hline
                17 & 17 & 13 & 23 & 19 & 5 & 1 & 11 & 7 \\ \hline
                19 & 19 & 23 & 13 & 17 & 7 & 11 & 1 & 5 \\ \hline
                23 & 23 & 19 & 17 & 13 & 11 & 7 & 5 & 1 \\ \hline
            \end{array}
        \]
    \end{enumerate}

    \item Let \(G\) be an abelian group. If \(x,y\in G\), then show that \((xy)^n = x^ny^n\) for any integer \(n\). (10 points)

    The case where \(n = 0\) is just \((xy)^0 = e\), and \(x^0y^0 = ee = e\), so \((xy)^n = x^ny^n\). We prove that \((xy)^n = x^ny^n\) holds for all positive integers \(n\). It is true in the base case \(n = 1\) since \((xy)^1 = xy\), and \(x^1y^1 = xy\). Suppose it is true for an arbitrary \(n\). Then,
    \begin{align*}
        (xy)^{n + 1} &= (xy)^n (xy) \\
        (xy)^{n + 1} &= x^ny^nxy \\
        (xy)^{n + 1} &= x^nxy^ny \\
        (xy)^{n + 1} &= x^{n+1}xy^{n+1}
    \end{align*}
    Hence, \((xy)^{n + 1} = x^{n+1}xy^{n+1}\) for all positive integers. And in the case where \(n\) is a negative integer,
    \begin{align*}
        (xy)^n &= \left[(xy)^{-1}\right]^{-n} \\
        (xy)^n &= \left(y^{-1}x^{-1}\right)^{-n} \\
        (xy)^n &= (y^{-1})^{-n}(x^{-1})^{-n} \\
        (xy)^n &= y^nx^n \\
        (xy)^n &= x^ny^n
    \end{align*}

    Therefore, if \(G\) is an abelian group, then \((xy)^n = x^ny^n\) for any integer \(n\) and for any \(x,y\in G\).

    \item Let \(G\) be a finite group with identity \(e\). Show that the set \(\{x\in G \,\mid\, x^{3} = e\}\) contains an odd number of elements.

    Let \(S = \{x\in G \,\mid\, x^{3} = e\}\), and let \(a\in S\) be arbitrary. Then, \(a\in G\), and \(a^3 = e\). Since \(a\in G\), then it satisfies the group axioms. We have \(a^3 = e\) which is equivalent to \(aa^2 = e\), so \(a^{-1} = a^2\). If \(a^2 \neq a\), then \(a^{-1} \neq a\) so for each \(a\in S\) where \(a^{2} \neq a\), another element, \(a^2\) will also be added. Hence, they come in pairs and collecting all of these would give us an even number. And if \(a^2 = a\), then \(aa^2 = aa = a^2 = e\). This implies \(a = e\), and so we only have one element to add. The identity is the only element we can add that is by itself, as having another one would imply another distinct identity element.

    Therefore, the number of elements in \(S\) is odd.

    \item Let \(G = \left\{\left.\begin{bmatrix}
        a & a \\ a & a
    \end{bmatrix} \right| a\in \real^*\right\}\). Prove that \(G\) is a group under matrix multiplication. (15 points)

    \(G\) is a group because it satisfies the group axioms:
    \begin{itemize}
        \item Closure

        For any \(A,B\in G\) such that \(A = \begin{bmatrix}
            a & a \\ a & a
        \end{bmatrix}\) and \(B = \begin{bmatrix}
            b & b \\ b & b
        \end{bmatrix}\),
        \begin{align*}
            AB &= \begin{bmatrix}
                a & a \\ a & a
            \end{bmatrix}\begin{bmatrix}
                b & b \\ b & b
            \end{bmatrix} \\
            AB &= \begin{bmatrix}
                ab + ab & ab + ab \\ ab + ab & ab + ab
            \end{bmatrix} \\
            AB &= \begin{bmatrix}
                2ab & 2ab \\ 2ab & 2ab
            \end{bmatrix}
        \end{align*}
        which we can see that \(AB\in G\).

        \item Associativity

        Matrix multiplication is associative, and since we are dealing with a specific type of matrix, then associativity is inherited.

        \item Identity

        We need to find an \(E\in G\) such that for all \(A\in G\), \(AE=A\).
        \begin{align*}
            AB &= E \\
            \begin{bmatrix}
                a & a \\ a & a
            \end{bmatrix}\begin{bmatrix}
                e & e \\ e & e
            \end{bmatrix} &= \begin{bmatrix}
                a & a \\ a & a
            \end{bmatrix} \\
            \begin{bmatrix}
                2ae & 2ae \\ 2ae & 2ae
            \end{bmatrix} &= \begin{bmatrix}
                a & a \\ a & a
            \end{bmatrix} \\
            2ae &= a \\
            e &= \frac{1}{2}.
        \end{align*}
        Since \(1/2 \in\real^*\), then \(E\) is the identity. To be sure, we can check that \(EA = A\):
        \begin{align*}
            EA &= A \\
            \begin{bmatrix}
                e & e \\ e & e
            \end{bmatrix}\begin{bmatrix}
                a & a \\ a & a
            \end{bmatrix} &= \begin{bmatrix}
                a & a \\ a & a
            \end{bmatrix} \\
            \begin{bmatrix}
                2ea & 2ea \\ 2ea & 2ea
            \end{bmatrix} &= \begin{bmatrix}
                a & a \\ a & a
            \end{bmatrix} \\
            2ea &= a \\
            e &= \frac{1}{2}.
        \end{align*}
        We got the same result, so \(E\) is our identity element.

        \item Inverse

        We need to find an element \(B\in G\), such that for all \(A\in G\), \(AB = E\):
        \begin{align*}
            AB &= E \\
            \begin{bmatrix}
                2ab & 2ab \\ 2ab & 2ab
            \end{bmatrix} &= \begin{bmatrix}
                \frac{1}{2} & \frac{1}{2} \\ \frac{1}{2} & \frac{1}{2}
            \end{bmatrix} \\
            2ab &= \frac{1}{2} \\
            b &= \frac{1}{4a}
        \end{align*}
        Since \(a\in\real^*\), then \(\frac{1}{4a}\in\real^*\). This means that \(B\) exists, and so the inverse of \(A\) exists.
    \end{itemize}
\end{enumerate}
