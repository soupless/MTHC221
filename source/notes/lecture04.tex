% Introduction to Groups, continued
% 01/29/2024

\section{Introduction to groups, continued}
    \renewcommand{\leftmark}{January 29, 2024}
    \begin{dfn}[Set of integers modulo \(n\)]
        We define \(\int_n\) be the set of integers modulo \(n\), such that \[\int_n = \{0, 1, 2, \ldots, n - 1\}.\] 

        Note that \(a \pmod{n}\) is the remainder when \(a\) is divided by \(n\).
    \end{dfn}

    \begin{example}
        Let \(*\) be the binary operation on \(\int_n\) defined by \(a * b = (a + b)\pmod{n}\). The table for \(*\) on \(\int_4\) is
        \[
            \begin{array}{c|c|c|c|c|}
                * & 0 & 1 & 2 & 3 \\ \hline
                0 & 0 & 1 & 2 & 3 \\ \hline
                1 & 1 & 2 & 3 & 0 \\ \hline
                2 & 2 & 3 & 0 & 1 \\ \hline
                3 & 3 & 0 & 1 & 2 \\ \hline
            \end{array}
        \]
    \end{example}

    \begin{thm}
        \(\langle \int_n, * \rangle\) is a group.
    \end{thm}

    \begin{proof}
        This is because it satisfies the group axioms:
        \begin{itemize}
            \item The operation \(*\) is a binary operation in \(\int_n\).
            \item The binary operation \(*\) is associative on \(\int_n\).
            \item The binary operation \(*\) has an identity element which is \(0\).
            \item If \(a\in\int_n\), then \(a^{-1} = n - a \pmod{n}\).
        \end{itemize}
    \end{proof}

    \begin{dfn}[Set of integers relatively prime to \(n\)]
        We define \(U_n\) to be the set of integers relatively prime to \(n\). That is, \[U_n = \{x \,\mid\, 1 \leq x \leq n \land \gcd(n, x) = 1\}.\]
    \end{dfn}

    \begin{thm}
        \(\langle U_n, *\rangle\) is a group.
    \end{thm}

    \begin{proof}
        \(\langle U_n, *\rangle\) satisfies the group axioms:
        \begin{itemize}
            \item \(U_n\) is closed under \(*\).

            Let \(a, b\in U_n\). Then, \(\gcd(a, n) = \gcd(b, n) = 1\). Suppose \(\gcd(ab, n) \neq 1\). Then, there exists an integer \(p\) such that \(p \mid n\) and \(p \mid ab\). Let \(d\) be such a number. Then, by Euclid's lemma, if \(p \mid ab\), then \(p \mid a\) or \(p \mid b\). Suppose \(p \mid a\). Then, \(\gcd(a, n) \geq p\) which contradicts our assumption that \(\gcd(a, n) = 1\). Similarly, if \(p \mid b\), then \(\gcd(b, n) \geq p\) which also contradicts our assumption that \(\gcd(b, n) = 1\). Regardless of the case, we have a contradiction.

            Hence, \(\gcd(ab, n) = 1\).

            \item \(*\) is associative.

            Multiplying integers in modular arithmetic is associative. Since \(U_n\) has a binary operation utilizing modular arithmetic, this means \(*\) is associative.

            \item \(U_n\) has an identity element under \(*\).

            Let \(a, e\in U_n\). Then, we must have \(ae = a\), or equivalently, \(ae \equiv a \pmod{n}\). Then, \(n \divs ae - a\) which is the same as \(n \divs a(e - 1)\). By Euclid's lemma, either \(n \divs a\) or \(n \divs e - 1\), which is the same as \(e \equiv 1 \pmod{n}\). Clearly, \(\gcd(1, n) = 1\), so our identity element is 1.

            \item Every element in \(U_n\) has an inverse.

            We are to find \(x\in U_n\) such that \(ax \equiv 1 \pmod{n}\) for every \(a\in U_n\). This is the same as finding \(x\) such that \(nk = ax - 1 \iff ax - nk = 1\). By \href{https://en.wikipedia.org/wiki/B%C3%A9zout%27s_identity}{Bezout's lemma \ExternalLink}, there exists an \(x\) satisfying the equation. Hence, \(a\) has an inverse in \(U_n\).
        \end{itemize}
    \end{proof}

    \begin{example}
        \mbox{}

        \begin{itemize}
            \item The table for \(\int_6\) is 
            \[
                \begin{array}{c|c|c|c|c|c|c|}
                    * & 0 & 1 & 2 & 3 & 4 & 5 \\ \hline
                    0 & 0 & 1 & 2 & 3 & 4 & 5 \\ \hline
                    1 & 1 & 2 & 3 & 4 & 5 & 0 \\ \hline
                    2 & 2 & 3 & 4 & 5 & 0 & 1 \\ \hline
                    3 & 3 & 4 & 5 & 0 & 1 & 2 \\ \hline
                    4 & 4 & 5 & 0 & 1 & 2 & 3 \\ \hline
                    5 & 5 & 0 & 1 & 2 & 3 & 4 \\ \hline
                \end{array}
            \]

            \item The table for \(U_9\) is
            \[
                \begin{array}{c|c|c|c|c|c|c|}
                    * & 1 & 2 & 4 & 5 & 7 & 8 \\ \hline
                    1 & 1 & 2 & 4 & 5 & 7 & 8 \\ \hline
                    2 & 2 & 4 & 8 & 1 & 5 & 7 \\ \hline
                    4 & 4 & 8 & 7 & 2 & 1 & 5 \\ \hline
                    5 & 5 & 1 & 2 & 7 & 8 & 4 \\ \hline
                    7 & 7 & 5 & 1 & 8 & 4 & 2 \\ \hline
                    8 & 8 & 7 & 5 & 4 & 2 & 1 \\ \hline
                \end{array}
            \]

            \item Let \(G = \{e, a, b, c, d\}\) where \(e\) is the identity in \(G\) under \(*\). Complete the table below: 

            \[
                \begin{array}{c|c|c|c|c|c|}
                    * & e & a & b & c & d \\ \hline
                    e & \color{blue}{e} & a & b & c & d \\ \hline
                    a & a & \color{blue}{b} & c & d & e \\ \hline
                    b & b & \color{blue}{c} & \color{blue}{d} & \color{blue}{e} & a \\ \hline
                    c & c & \color{blue}{d} & e & \color{blue}{a} & \color{blue}{b} \\ \hline
                    d & d & e & a & b & c \\ \hline
                \end{array}
            \]
        \end{itemize}
    \end{example}

    \begin{example}
        Define \(*\) on \(\real\) by \(a * b = a + b + ab\). Verify if \(\langle \real, * \rangle\) is a group.

        We can see that \(a + b + ab = (a + 1)(b + 1) - 1\). Since \(a,b\in\real\), then \(a * b\in\real\). We also know that \(*\) \hyperref[exm:starAssoc]{is associative}. Consider \(a * e = a\). Solving for the identity,
        \begin{align*}
            a + e + ae &= a \\
            e + ae &= 0 \\
            e(1 + a) &= 0
        \end{align*}
        This means \(e = 0\). To check if every element in \(\real\) has an inverse under \(*\), suppose \(b = a^{-1}\). Then, \(a * b = 0\). Solving for \(b\) in terms of \(a\),
        \begin{align*}
            a + b + ab &= 0 \\
            b + ab &= -a \\
            b(1 + a) &= -a \\
            b &= \frac{-a}{1 + a}
        \end{align*}
        We can see that \(b\) exists only if \(a \neq 1\). Hence, not all elements have an inverse, so \(\langle \real, * \rangle\) is not a group. 
    \end{example}

    \begin{note}
        Let \(G\) be a group. For any \(a, b\in G\), we define \(ab\) as \(a*b\) where \(*\) is the binary operation of \(G\), if \(*\) is not stated.
    \end{note}

    \marker{thm:unique-identity}
    \begin{thm}
        The identity element in a group is unique.
    \end{thm}

    \begin{proof}
        Suppose \(e_1\) and \(e_2\) are both identities of a group. Then, \(e_1e_2 = e_1\) since \(e_2\) is an identity. Similarly, \(e_1e_2 = e_2\) since \(e_1\) is an identity. By transitivity, we have \(e_1 = e_2\).
    \end{proof}

    \marker{thm:cancellation-law}
    \begin{thm}
        Let \(a, b, c\in G\) where \(G\) is a group. Then,
        \begin{enumerate}
            \item[(i)] \(ac = bc \implies a = b\). This is called right cancellation.
            \item[(ii)] \(ca = cb \implies a = b\). This is called left cancellation.
        \end{enumerate}
    \end{thm}

    \begin{proof}
        Let \(a, b, c\in G\) be arbitrary. We prove each item:
        \begin{enumerate}
            \item[(i)] Suppose that \(ac = bc\). Then, \(acc^{-1} = bcc^{-1}\). Since the binary operation in \(G\) is associative, we have \(a(cc^{-1}) = b(cc^{-1})\), which simplifies to \(a = b\).

            \item[(ii)] Suppose that \(ca = cb\). Then, \(c^{-1}ca = c^{-1}cb\). Since the binary operation in \(G\) is associative, we have \((c^{-1}c)a = (c^{-1}c)b\), which simplifies to \(a = b\).
        \end{enumerate}
    \end{proof}

    \marker {thm:unique-inverse}
    \begin{thm}
        Let \(G\) be a group. The inverse of any element in \(G\) is unique.
    \end{thm}

    \begin{proof}
        Let \(g\in G\) be arbitrary. Let \(a, b\in G\) be inverses of \(g\). Then, \(ag = ga = e\), and \(bg = gb = e\), where \(e\) is the identity in \(G\). Consider \(agb\). We have \(a(gb) = b\), and \((ag)b = a\). Since \(G\) is associative, then \(a = b\).
    \end{proof}

    \begin{thm}
        Let \(G\) be a group and let \(a \in G\). Then, \((a^{-1})^{-1} = a\).
    \end{thm}

    \begin{proof}
        Since the inverse of \(a\) is \(a^{-1}\) and by the \hyperref[thm:unique-inverse]{uniqueness of inverses}, the inverse of \(a^{-1}\), \((a^{-1})^{-1}\), is \(a\).
    \end{proof}

    \begin{thm}
        Let \(G\) be a group. If \(a, b\in G\), then \((ab)^{-1} = b^{-1}a^{-1}\).
    \end{thm}

    \begin{proof}
        Let \(G\) be a group, and let \(a,b\in G\) be arbitrary. Then,
        \begin{align*}
            ab(ab)^{-1} &= e \\
            a^{-1}ab(ab)^{-1} &= a^{-1}e \\
            eb(ab)^{-1} &= a^{-1} \\
            b^{-1}eb(ab)^{-1} &= b^{-1}a^{-1} \\
            b^{-1}b(ab)^{-1} &= b^{-1}a^{-1} \\
            (ab)^{-1} &= b^{-1}a^{-1}
        \end{align*}
    \end{proof}

    \marker{thm:general-product-inverse}
    \begin{thm}
        Let \(G\) be a group. Let \(a_1, a_2, \ldots, a_n\in G\). Then,
        \((a_1a_2\cdots a_n)^{-1} = a_{n}^{-1} a_{n - 1}^{-1}\cdots a_{2}^{-1}a_{1}^{-1}\).
    \end{thm}

    \begin{proof}
        Let \(n\in \nat\) be arbitrary, and suppose that for all \(i\in\nat\) less than \(n\), \((a_1a_2\cdots a_i)^{-1} = a_{i}^{-1} a_{i - 1}^{-1}\cdots a_{2}^{-1}a_{1}^{-1}\).
        \begin{align*}
            (a_1a_2\cdots a_n)^{-1} = a_{n}^{-1} (a_1a_2\cdots a_{n - 1})^{-1} \\
            (a_1a_2\cdots a_n)^{-1} = a_{n}^{-1} a_{n - 1}^{-1}\cdots a_{2}^{-1}a_{1}^{-1}
        \end{align*}

        Hence, \((a_1a_2\cdots a_n)^{-1} = a_{n}^{-1} a_{n - 1}^{-1}\cdots a_{2}^{-1}a_{1}^{-1}\).
        \end{proof}
