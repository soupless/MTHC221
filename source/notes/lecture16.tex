\section{Properties of homomorphisms, and isomorphisms}
    \renewcommand{\leftmark}{May 13, 2024}

    \begin{thm}
        Let \(\phi : G \to G'\) be a group homomorphism, and \(g\in G\). Then, 
        \begin{enumerate}
            \item \(\phi(e) = e'\) where \(e\) and \(e'\) are the identity elements of \(G\) and \(G'\), respectively,
            \item \(\phi(g^n) = \phi(g)^n\) for all integers \(n\),
            \item If \(|g|\) is finite, then \(|\phi(g)|\) divides \(|g|\),
            \item \(\ker \phi \leq G\),
            \item \(\phi(a) = \phi(b)\) iff \(a\ker\phi = b\ker\phi\),
            \item If \(\phi(g) = g'\), then \(\phi^{-1}(g') = g\ker\phi\) where \(\phi^{-1}(g') = \{x\in G\,\mid\, \phi(x) = g'\}\).
        \end{enumerate}
    \end{thm}

    \begin{proof}
        We prove each item as follows:
        \begin{enumerate}
            \item Let \(g\in G\) be arbitrary. We consider \(\phi(ge)\) as follows:
            \begin{align*}
                \phi(ge) &= \phi(g) \\
                \phi(g)\phi(e) &= \phi(g) \\
                \left(\phi(g)\right)^{-1}\phi(g)\phi(e) &= \left(\phi(g)\right)^{-1}\phi(g) \\
                \phi(e) &= e'.
            \end{align*}

            \item We first prove that \(\phi(g^{-1}) = \phi(g)^{-1}\). We consider \(\phi(gg^{-1})\) as follows:
            \begin{align*}
                \phi(gg^{-1}) &= \phi(e) \\
                \phi(g)\phi(g^{-1}) &= e' \\
                \left(\phi(g)\right)^{-1}\phi(g)\phi(g^{-1}) &= \phi(g)^{-1} \\
                \phi(g^{-1}) &= \phi(g)^{-1}
            \end{align*}

            There are two cases for \(n\), and the first one is if \(n \geq 0\). Then,
            \begin{align*}
                \phi(g^n) &= \phi(\underbrace{g\cdot g \cdots g}_{n \text{ times}}) \\
                \phi(g^n) &= \underbrace{\phi(g) \cdot \phi(g) \cdots \phi(g)}_{n \text{ times}} \\
                \phi(g^n) &= \phi(g)^n
            \end{align*}
            And if \(n < 0\), then
            \begin{align*}
                \phi(g^n) &= \phi(g^{-|n|}) \\
                \phi(g^n) &= \phi((g^{-1})^{|n|}) \\
                \phi(g^n) &= \phi(g^{-1})^{|n|} \\
                \phi(g^n) &= (\phi(g)^{-1})^{|n|} \\
                \phi(g^n) &= \phi(g)^{-|n|} \\
                \phi(g^n) &= \phi(g)^{n}
            \end{align*}

            \item Let \(|g| = n\). Then, \(g^n = e\). It follows that
            \begin{align*}
                \phi(g^n) &= \phi(e) \\
                \phi(g)^n &= e'
            \end{align*}
            Hence, \(|\phi(g)|\) divides \(n\), so \(|\phi(g)|\) divides \(|g|\).

            \item Let \(a,b\in \ker\phi\). Then,
            \begin{align*}
                \phi(ab^{-1}) &= \phi(a)\phi(b^{-1}) \\
                \phi(ab^{-1}) &= e'e' \\
                \phi(ab^{-1}) &= e'
            \end{align*}
            Hence, \(ab^{-1}\in\ker\phi\). By the one-step subgroup test, \(\ker\phi \leq G\).

            \item Suppose \(\phi(a) = \phi(b)\). Then,
            \begin{align*}
                \phi(a) &= \phi(b) \\
                \phi(b)^{-1}\phi(a) &= \phi(b)^{-1}\phi(b) \\
                \phi(b^{-1})\phi(a) &= e' \\
                \phi(b^{-1a}) &= e' \\
                b^{-1}a &\in \ker\phi \\
                b^{-1}a \ker\phi &= \ker\phi \\
                bb^{-1}a \ker\phi &= b\ker\phi \\
                a \ker\phi &= b\ker\phi
            \end{align*}
            The backward direction can be proven by simply doing the proof of the forward direction in reverse.

            \item Suppose \(\phi(g) = g'\). Let \(y\in \phi^{-1}(g')\) be arbitrary. Then,
            \begin{align*}
                y &\in \phi^{-1}(g') \\
                \phi(y) &= g' \\
                \phi(y) &= \phi(g) \\
                \phi(g)^{-1}\phi(y) &= e' \\
                \phi(g^{-1})\phi(y) &= e' \\
                \phi(g^{-1}y) &= e' \\
                g^{-1}y &\in \ker\phi \\
                g^{-1}y \ker\phi &= \ker\phi \\
                y \ker\phi &= g\ker\phi
            \end{align*}
            Since \(y\in y\ker\phi\), this implies \(y\in g\ker\phi\). Hence, \(\phi^{-1}(g') \subseteq g\ker\phi\).

            Now, let \(z\in g\ker\phi\) be arbitrary. Then,
            \begin{align*}
                z &= ga \qquad (\exists a\in\ker\phi) \\
                \phi(z) &= \phi(ga) \\
                \phi(z) &= \phi(g)\phi(a) \\
                \phi(z) &= g'e' \\
                \phi(z) &= g'
            \end{align*}
            which means \(z\in \phi^{-1}(g')\). Hence, \(g\ker\phi \in \phi^{-1}(g')\), and so \(\phi^{-1}(g') = g\ker\phi\).
        \end{enumerate}
    \end{proof}

    The last item gives us information about which elements also map to \(g'\), provided we know at least one element \(g\in G\).

    \begin{note}
        Given a homomorphism \(\phi : G \to G'\) and \(H \leq G\), we define \(\phi(H)\) as \(\{\phi(h) \,\mid\, h\in H\}\).
    \end{note}

    \begin{thm}
        Let \(\phi : G \to G'\) be a group homomorphism and let \(H \leq G\). Then,
        \begin{enumerate}
            \item \(\phi(H) \leq G'\).
            \item If \(H\) is cyclic, then \(\phi(H)\) is cyclic.
            \item If \(H\) is abelian, then \(\phi(H)\) is abelian.
            \item If \(H\) is normal, then \(\phi(H)\) is normal.
            \item If \(|\ker\phi| = n\), then \(\phi\) is an \(n\)-to-1 mapping from \(G\) onto \(\phi(G)\).
            \item If \(|H| = n\), then \(|\phi(H)|\) divides \(n\).
        \end{enumerate}
    \end{thm}

    \begin{proof}
        \mbox{}

        \begin{enumerate}
            \item Checking that \(\phi(H)\) is nonempty is trivial, since the identity element of \(H\) is mapped onto \(\phi(H)\). Let \(h_1, h_2 \in H\). Then,
            \begin{align*}
                \phi(h_1)\phi(h_2)^{-1} &= \phi(h_1)\phi(h_2^{-1}) \\
                \phi(h_1)\phi(h_2)^{-1} &= \phi(h_1h_2^{-1})
            \end{align*}
            Since \(h_1h_2^{-1} \in H\), then \(\phi(h_1h_2^{-1} \in \phi(H))\). By the one-step subgroup test, \(\phi(H) \leq G\).

            \item Suppose \(H\) is cyclic. Then, there is an element \(a\in H\) such that \(H = \langle a \rangle\). We claim that \(\phi(a)\) generates \(\phi(H)\). Let \(h\in H\) such that \(\phi(h) \in \phi(H)\). Since \(H\) is cyclic, we can find an integer \(k\) such that \(h = a^k\). Then,
            \begin{align*}
                h &= a^k \\
                \phi(h) &= \phi(a^k) \\
                \phi(h) &= \phi(a)^k \\
                \phi(H) &\leq \langle \phi(a)\rangle.
            \end{align*}
            Clearly, \(\langle \phi(a)\rangle \leq \phi(H)\). Hence, \(\phi(H) = \langle \phi(a)\rangle\), so \(\phi(H)\) is cyclic.

            \item Suppose \(H\) is abelian. Let \(h_1, h_2\in H\). Then,
            \begin{align*}
                \phi(h_1)\phi(h_2) &= \phi(h_1h_2) \\
                \phi(h_1)\phi(h_2) &= \phi(h_2h_1) \\
                \phi(h_1)\phi(h_2) &= \phi(h_2)\phi(h_1)
            \end{align*}
            Hence, \(\phi(H)\) is abelian.

            \item Suppose \(H\) is normal in \(G\). Let \(h\in H\), \(g\in G\). We consider \(\phi(g)\phi(h)\phi(g)^{-1}\):
            \begin{align*}
                \phi(g)\phi(h)\phi(g)^{-1} &= \phi(g)\phi(h)\phi(g^{-1}) \\
                \phi(g)\phi(h)\phi(g)^{-1} &= \phi(ghg^{-1})
            \end{align*}
            Since \(H\) is normal, there is an \(h_1\in H\) such that \(h_1 = ghg^{-1}\). Then,
            \begin{align*}
                \phi(g)\phi(h)\phi(g)^{-1} &= \phi(h_1) \\
                \phi(g)\phi(h)\phi(g)^{-1} &\in \phi(H)
            \end{align*}
            Therefore, \(\phi(H)\) is normal in \(\phi(G)\).

            \item If \(\phi(g) = g'\), then \(\phi^{-1}(g') = g\ker\phi\). Since \(|g\ker\phi| = |\ker\phi| = n\), then there are \(n\) elements mapped to \(g'\) where \(g' \in \phi(G)\).

            \item Trivial.
        \end{enumerate}
    \end{proof}

    \begin{thm}
        Let \(\phi : G \to G'\) be a homomorphism of groups. Then, \(\ker\phi \normalsubeq G\).
    \end{thm}

    \begin{proof}
        Let \(g\in G\) and \(x \in \ker\phi\). Consider \(gxg^{-1}\):
        \begin{align*}
            \phi(gxg^{-1}) &= \phi(g)\phi(x)\phi(g^{-1}) \\
            \phi(gxg^{-1}) &= \phi(g)e'\phi(g)^{-1} \\
            \phi(gxg^{-1}) &= \phi(g)\phi(g)^{-1} \\
            \phi(gxg^{-1}) &= e'
        \end{align*}
        Hence, \(gxg^{-1} \in \ker\phi\), so \(\ker\phi \normalsubeq G\).
    \end{proof}

    \begin{dfn}[Isomorphism]
        Let \(\phi : G \to G'\) be a homomorphism of groups. If \(\phi\) is bijective, then \(\phi\) is called an isomorphism.
    \end{dfn}

    \begin{example}
        Let \(G\) be an abelian group. Let \(\theta : G \to G\) where \(\theta(g) = g^{-1}\). Then,
        \begin{align*}
            \theta(g_1g_2) &= (g_1g_2)^{-1} \\
            \theta(g_1g_2) &= g_2^{-1}g_1^{-1} \\
            \theta(g_1g_2) &= g_1^{-1}g_2^{-1} \\
            \theta(g_1g_2) &= \theta(g_1)\theta(g_2)
        \end{align*}
        So \(\theta\) is a homomorphism. We can see that \(\theta\) is injective since
        \begin{align*}
            \theta(g_1) &= \theta(g_2) \\
            g_1^{-1} &= g_2^{-1} \\
            (g_1^{-1})^{-1} &= (g_2^{-1})^{-1} \\
            g_1 &= g_2
        \end{align*}
        And it is also surjective. Choose any \(g\in G\). We can find an element in \(G\) that \(\theta\) maps to \(g\), and this is \(g^{-1}\): \(\theta(g^{-1}) = (g^{-1})^{-1} = g\).
    \end{example}