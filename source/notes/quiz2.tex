% Mess I made because I don't know moving arguments

\renewcommand{\sectionmark}[1]{\markboth{}{Quiz \thequizcounter}}
\titleformat{\section}{\Large\bfseries}{\phantomsection Quiz~\thequizcounter}{1ex}{}
\section[Quiz \thequizcounter]{}\normalsize\titleformat{\section}{\Large\bfseries}{\phantomsection Lecture~\thesection:}{1ex}{}
\addtocounter{section}{-1}
\renewcommand{\sectionmark}[1]{\markboth{}{Lecture \thesection: #1}}
% \vspace{-0.2in}
\renewcommand{\leftmark}{March 04, 2024}

% Quiz #2
% 03/04/2024

\begin{enumerate}
    \item Determine the order of all elements in the following groups. (10 points)

    \begin{enumerate}
        \item[(i)] \(\int_8\)

        We just solve for the order of each element by repeatedly applying the operation. The operation application is omitted. Hence, \(|0| = 1\), \(|1| = 8\), \(|2| = 4\), \(|3| = 8\), \(|4| = 2\), \(|5| = 8\), \(|6| = 4\), \(|7| = 8\).

        \item[(ii)] \(U_{10}\)

        A similar process can be done for this item, just repeated application of the binary operation of \(U_{10}\). Hence, \(|1| = 1\), \(|3| = 4\), \(|7| = 4\), \(|9| = 2\).
    \end{enumerate}

    \item Define \(SL(n, \real)\) be the set of \(n \times n\) matrices with real entries whose determinant is 1. Show that \(SL(n, \real) < GL(n, \real)\) by using the two-step subgroup test. (10 points)

    It is trivial that \(SL(n, \real) \subset GL(n, \real)\) since given any arbitrary \(x\in SL(n, \real)\), \(\det(x) = 1 \neq 0\), so \(x \in GL(n, \real)\). However, not all elements of \(GL(n, \real)\) are in \(SL(n, \real)\), i.e., \(aI_n\) for \(a \in \real \backslash \{0, 1\}\). 

    To show closure, let \(a,b\in SL(n, \real)\). Then, \(\det(a) = 1\) and \(\det(b) = 1\). Consider \(ab\):
    \begin{align*}
        \det(ab) &= \det(a) \det(b) \\
        \det(ab) &= 1\cdot 1 \\
        \det(ab) &= 1
    \end{align*}
    This means that \(ab\in SL(n, \real)\). It can be easily shown that inverses exist simply by \(\det(a^{-1}) = \det(a)^{-1} = 1^{-1} = 1\), so \(a^{-1} \in SL(n, \real)\).

    By the two-step subgroup test, this means that \(SL(n, \real) < GL(n, \real)\).

    \item Let \(G\) be a group and let \(H, K\) be subgroups of \(G\). Show that \(H \cap K \leq G\). (10 points)

    Let \(a, b\in H \cap K\) be arbitrary. Then, \(a,b \in H\), and \(a,b \in K\). Since both \(H\) and \(K\) are subgroups, then \(ab \in H\), and \(ab \in K\), so \(ab \in H \cap K\). Hence, \(H \cap K\) is closed. Also, since \(a \in H\) and \(a \in K\), this means \(a^{-1} \in H\) and \(a^{-1} \in K\), so \(a^{-1} \in H \cap K\). Hence, inverses exist.

    By the two-step subgroup test, this means that \(H \cap K \leq G\).

    \item Let \(a, b\) be group elements such that \(|a| = 4\), \(|b| = 2\), and \(a^3 b = ba\). Find \(|ba|\). (15 points)

    This means that \(b^2 = e\), \(a^4 = e\). Then,
    \begin{align*}
        b^2 a^4 &= e \\
        b^2 aa^3 b &= b \\
        b^2 aba &= b \\
        baba &= e \\
        (ba)^2 &= e
    \end{align*}
    This means that \(|ba| \leq 2\). Suppose \(|ba| = 1\). Then, \(ba = e\), so \(a^{-1} = b\) and \(a^3 b = a^3 a^{-1} = a^2 = e\) which implies that \(|a| \leq 2\), contradicting our assumption that \(|a| = 4\). Therefore, \(|ba| = 2\).

    \item Let \(x\) be a group element such that \(|x| = 5\). Prove that \(C(x) = C(x^3)\). (15 points)

    Let \(a \in C(x)\) be arbitrary. Then,
    \begin{align*}
        ax &= xa \\
        x^2ax &= x^2xa \\
        xxax &= x^3 a \\
        xax^2 &= x^3 a \\
        ax^3 &= x^3 a
    \end{align*}
    Hence, \(a\in C(x^3)\), and so \(C(x) \subseteq C(x^3)\). Now, consider \(b \in C(x^3)\):
    \begin{align*}
        bx^3 &= x^3b \\
        x^3bx^3 &= x^3x^3b \\
        bx^3x^3 &= x^5xb \\
        bxx^5 &= xb \\
        bx &= xb
    \end{align*}
    So, \(b\in C(x)\) which implies \(C(x^3) \subseteq C(x)\). Therefore, \(C(x) = C(x^5)\).
\end{enumerate}