\section{Order and subgroups, continued}
    \renewcommand{\leftmark}{February 26, 2024}

    \begin{example}
        Let \(G\) be an abelian group. and let \(H,K \subseteq G\) be subgroups. Show that the set \(HK := \big\{hk \,\big|\,h\in H, k\in K\big\} \subseteq G\).

        We prove this claim using the one-step subgroup test. Let \(a,b\in HK\). Then, \(a = h_1k_1\) and \(b = h_2k_2\), so
        \begin{align*}
            ab^{-1} &= h_1k_1(h_2k_2)^{-1} \\
            ab^{-1} &= h_1k_1h_2^{-1}k_2^{-1} \\
            ab^{-1} &= h_1h_2^{-1}k_1k_2^{-1}
        \end{align*}

        Since \(h_1h_2^{-1} \in H\), and \(k_1k_2^{-1} \in K\), then \(ab^{-1} \in HK\).
    \end{example}

    \begin{thm}[Finite subgroup test]
        Let \(H\) be a nonempty finite subset of a group \(G\). If \(H\) is closed, then \(H \subseteq G\).
    \end{thm}

    \begin{proof}
        By hypothesis, \(H \neq \emptyset\) and \(H\) is closed. We need to show that for all \(h\in H\), \(h^{-1} \in H\). Suppose \(H\) is finite. Let \(h\in H\) be arbitrary. Consider the following cases:
        \begin{enumerate}
            \item[Case 1:]\(h = e\).

            Hence, \(h^{-1} = e^{-1} = e\).

            \item[Case 2:]\(h \neq e\).

            Consider the set \(T = \{h^k \,|\, k\in\int^+\}\). If each power of \(h\) is unique, then \(T\) is an infinite set. However, we know that \(H\) is finite, hence, a contradiction. Therefore, \(T\) is a finite set.

            This means that there exists \(i, j\in \int^+\) such that \(h^i = h^j\) and \(i \neq j\). Without loss of generality, let \(i > j\). Since \(h^k \in H\) for all \(k\in\int\), then \(T \subseteq H\). We then have
            \begin{align*}
                h^i &= h^j \\
                h^i h^{-j} &= h^j h^{-j} \\
                h^{i-j} &= e
            \end{align*}

            We claim that \(i - j \geq 2\). Suppose that \(i - j < 2\). If \(i - j = 0\), then \(i = j\) which contradicts our proof that \(i \neq j\). If \(i - j = 1\), then \(h^{i - j} = h^1 = h = e\), which contradicts our assumption that \(h \neq e\). Hence, \(i - j \geq 2\).

            Then,
            \begin{align*}
                h^{i - j} &= e \\
                hh^{i - j - 1} &= e
            \end{align*}
            This means that \(h^{-1} = h^{i - j - 1}\). Hence, the inverse of \(h\) exists.

            By the two-step subgroup test, \(H \leq G\).
        \end{enumerate}
    \end{proof}

    % | INFO: Subgroup is \leq, subset is \subseteq

    \begin{dfn}[Subgroup generated by an element]
        Let \(g\in G\) where \(G\) is a group. Then, \(\langle g \rangle\) is the subgroup generated by \(g\), and is defined as \(\{g^k \,|\, k\in\int\}\).
    \end{dfn}

    \begin{dfn}[Centralizer]
        Let \(G\) be a group, and let \(a\in G\). Then, \(C(a)\) is the centralizer of \(a\), defined as the set \(\{x\,|\, x\in G \land ax = xa\}\).
    \end{dfn}

    \begin{example}
        Let \(G\) be a group, and let \(a\in G\). Show that \(C(a) \leq G\).

        \begin{itemize}
            \item We can use the two-step subgroup test. It is obvious that \(C(a)\) is nonempty, since \(e\in C(a)\). To prove closure, let \(x,y\in C(a)\) be arbitrary. We then have \(axy = xay = xya\), which implies \(xy\in C(a)\). And to prove that inverses exist, 
            \begin{align*}
                ax &= xa \\
                x^{-1}axx^{-1} &= x^{-1}xax^{-1} \\
                x^{-1}a &= ax^{-1}
            \end{align*}
            which implies \(x^{-1}\in C(a)\). By the two-step subgroup test, this means \(C(a) \leq G\).
        \end{itemize}
    \end{example}