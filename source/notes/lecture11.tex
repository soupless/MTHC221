\section{Permutation group}
    \renewcommand{\leftmark}{April 03, 2024}

    \begin{dfn}[Permutation of a set]
        A permutation of a set \(S\) is a bijective function from \(S\) to \(S\).
    \end{dfn}

    \begin{example}
        Let \(S = \{1, 2, 3\}\). Then, permutations \(\sigma_i : S \to S\) would be \begin{align*}
            \sigma_1 &= \begin{bmatrix}
                1 & 2 & 3 \\ 
                1 & 3 & 2
            \end{bmatrix} &
            \sigma_2 &= \begin{bmatrix}
                1 & 2 & 3 \\
                1 & 2 & 3
            \end{bmatrix}& 
            \sigma_3 &= \begin{bmatrix}
                1 & 2 & 3 \\
                3 & 1 & 2
            \end{bmatrix} \\
            \sigma_4 &= \begin{bmatrix}
                1 & 2 & 3 \\ 
                3 & 2 & 1
            \end{bmatrix} &
            \sigma_5 &= \begin{bmatrix}
                1 & 2 & 3 \\
                2 & 3 & 1
            \end{bmatrix}& 
            \sigma_3 &= \begin{bmatrix}
                1 & 2 & 3 \\
                2 & 1 & 3
            \end{bmatrix}
        \end{align*}
    \end{example}

    \begin{thm}
        Let \(n = |S|\). The number of permutations of a set \(S\) is \(n!\).
    \end{thm}

    \begin{thm}
        If \(\sigma\) is a permutation of \(S\), then \(\sigma \cdot \id = \id \cdot \sigma = \sigma\) where \(\id\) is the identity permutation.
    \end{thm}

    \begin{dfn}[Symmetric group of degree \(n\)]
        Let \(S = \{1, 2, \ldots, n\}\). We define \(S_n\) to be the set of permutations of a set with \(n\) elements. This is called the symmetric group of degree \(n\).
    \end{dfn}

    \begin{thm}
        \(S_n\) is a group.
    \end{thm}

    \begin{proof}
        We prove that \(S_n\) satisfies the group axioms:
        \begin{itemize}
            \item Closure: \(\sigma, \tau \in S_n \implies \sigma\tau \in S_n\).
            \item Associativity: Trivial.
            \item Identity: Define \(\id : S \to S\) by \(\id(i) = i\) for all \(i\in S\). Then \(\id\) is the identity in \(S_n\).
            \item Inverse: If \(\sigma\in S_n\), we let \(\sigma^{-1}\) be the mapping such that \(\sigma(i) = j\) iff \(\sigma^{-1}(j) = i\). 
        \end{itemize}
    \end{proof}

    \begin{dfn}[Identity mapping]
        We define \(\id : S \to S\) such that \(\id(i) = i\).
    \end{dfn}

    \begin{cor}
        If \(\sigma\) is a permutation in \(S\), then \(\sigma \cdot \id = \id \cdot \sigma = \sigma\).
    \end{cor}

    \begin{proof}
        Let \(i\in S\). Then,
        \begin{align*}
            \sigma \cdot \id(i) &= \sigma (i) \\
            \id \cdot \sigma(i) &= \sigma (i)
        \end{align*}
        Hence, \(\sigma \cdot \id = \id \cdot \sigma = \sigma\).
    \end{proof}

    \begin{dfn}[Dihedral group of degree \(n\)]
        The group of symmetries of a regular \(n\)-gon is called the dihedral group of degree \(n\).
    \end{dfn}

    The cardinality of \(D_n\) is \(2n\), since we have \(n\) turns and \(n\) flips.