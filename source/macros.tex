\usepackage{
    titlesec,
    hyperref,
    enumitem,
    amsmath, amsthm, amssymb,
    thmbox, 
    geometry,
    fancyhdr,
    % marginnote,
    indentfirst,
    parskip,
    setspace,
    wrapfig,
    tikz
}

\usepackage{graphicx}

\allowdisplaybreaks
\onehalfspacing

\geometry{
    paper=a4paper,
    outer=0.8in,
    top=1.1in,
    bottom=1.1in,
    inner=0.9in,
    bindingoffset=4mm
} 

\hypersetup{
    colorlinks,
    pdftitle=Abstract Algebra A,
    pdfauthor=soupless,
}

\pagestyle{fancy}

\setlength{\headheight}{13.6pt}
\fancyhead[RO, LE]{\textrm{\rightmark}}
\fancyhead[LO, RE]{\leftmark}
\renewcommand{\headrulewidth}{0.75pt}
\renewcommand{\footrulewidth}{0.75pt}

\renewcommand{\sectionmark}[1]{\markboth{}{Lecture \thesection: #1}}

\titleformat{\section}{\Large\bfseries}{Lecture~\thesection:}{1ex}{}

\setlength{\parindent}{0.25in}
\setlength{\parskip}{10pt}

\renewcommand{\emph}{\textsl}

\newenvironment{recall}
  {\begin{example}[Recall]}
  {\end{example}}

\counterwithin{figure}{section}

\thmboxoptions{cut=true, bodystyle=\normalfont, leftmargin=7mm, rightmargin=5mm, headstyle={#1 #2 \normalfont}, titlestyle={ (#1)}, vskip=1.5mm}


\newtheorem[M]{thm}{Theorem}[section]
\newtheorem[L]{dfn}{Definition}[section]
\newtheorem[S]{note}{Note}[section]
\newtheorem[S]{cor}{Corollary}[section]


\renewcommand{\iff}{\Leftrightarrow}

\newcommand{\oldint}{\int}
\renewcommand{\int}{\mathbb{Z}}
\newcommand{\nat}{\mathbb{N}}
\newcommand{\divs}{\,\mid\,}
\newcommand{\pset}[1]{\mathcal{P}\left(#1\right)}
\newcommand{\real}{\mathbb{R}}
\newcommand{\rat}{\mathbb{Q}}

\DeclareMathOperator{\diam}{diam}

\newcommand{\marker}[1]{\phantomsection\label{#1}}

\newcommand{\ExternalLink}{%
    \tikz[x=1.2ex, y=1.2ex, baseline=-0.05ex]{% 
        \begin{scope}[x=1ex, y=1ex]
            \clip (-0.1,-0.1) 
                --++ (-0, 1.2) 
                --++ (0.6, 0) 
                --++ (0, -0.6) 
                --++ (0.6, 0) 
                --++ (0, -1);
            \path[draw, magenta,
                line width = 0.5, 
                rounded corners=0.5] 
                (0,0) rectangle (1,1);
        \end{scope}
        \path[draw, magenta, line width = 0.5] (0.5, 0.5) 
            -- (1, 1);
        \path[draw, magenta, line width = 0.5] (0.6, 1) 
            -- (1, 1) -- (1, 0.6);
        }
    }

\newcounter{quizcounter}
\renewcommand\thequizcounter{\arabic{quizcounter}}
